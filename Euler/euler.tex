\documentclass[fleqn]{article}

\usepackage{amsmath}
\usepackage[top=2mm, bottom=2mm, left=2mm, right=30mm]{geometry}
\usepackage{forloop}
\usepackage{color}

\begin{document}

Problem 64.  Continued fractions
[2, 1, 3, 1, 2, 8]

\begin{flalign}
\begin{tabular}{*{20}{c|}}
 \textbf{Finding...} & \textbf{Step 1} &  \textbf{Step 2} &  \textbf{Step 3} \\
\hline
$\sqrt{19}$ : & $\sqrt{19} = {\color{red}4} + \frac{1}{x_1}$ & $x_1 = \frac{1}{\sqrt{19} - 4}$ & $ = \frac{1}{\sqrt{19} - 4} \frac{\sqrt{19} + 4}{\sqrt{19} + 4} = \frac{\sqrt{19} + 4}{19-16} = \frac{\sqrt{19} + 4}{3} $ &  \\
\hline
$x_1 = $ & $\frac{\sqrt{19} + 4}{3} = {\color{red}2} + \frac{1}{x_2}$ & $x_2 = \frac{6}{\sqrt{19} - 2}$ & $ = \frac{3}{\sqrt{19} - 2} \frac{\sqrt{19} + 2}{\sqrt{19} + 2} = \frac{3(\sqrt{19} + 2)}{19-4} = \frac{\sqrt{19} + 2}{5} $ & \\
\hline
$x_2 = $ & $\frac{\sqrt{19} + 2}{5} = {\color{red}1} + \frac{1}{x_3}$ & $x_3 = \frac{5}{\sqrt{19} - 3}$ & $ = \frac{5}{\sqrt{19} - 3} \frac{\sqrt{19} + 3}{\sqrt{19} + 3} = \frac{5(\sqrt{19} + 3)}{19-9} = \frac{\sqrt{19} + 3}{2} $ & \\
\hline
$x_3 = $ & $\frac{\sqrt{19} + 3}{2} = {\color{red}3} + \frac{1}{x_4}$ & $x_4 = \frac{2}{\sqrt{19} - 3}$ & $ = \frac{2}{\sqrt{19} - 3} \frac{\sqrt{19} + 3}{\sqrt{19} + 3} = \frac{2(\sqrt{19} + 3)}{19-9} = \frac{\sqrt{19} + 3}{5} $ & \\
\hline
$x_4 = $ & $\frac{\sqrt{19} + 3}{5} = {\color{red}1} + \frac{1}{x_5}$ & $x_5 = \frac{5}{\sqrt{19} - 2}$ & $ = \frac{5}{\sqrt{19} - 2} \frac{\sqrt{19} + 2}{\sqrt{19} + 2} = \frac{5(\sqrt{19} + 2)}{19-4} = \frac{\sqrt{19} + 2}{3} $ & \\
\hline
$x_5 = $ & $\frac{\sqrt{19} + 2}{3} = {\color{red}2} + \frac{1}{x_6}$ & $x_6 = \frac{5}{\sqrt{19} - 3}$ & $ = \frac{3}{\sqrt{19} - 4} \frac{\sqrt{19} + 4}{\sqrt{19} + 4} = \frac{3(\sqrt{19} + 4)}{19-16} = \sqrt{19} + 4 $ & \\
\hline
\end{tabular}
\end{flalign}

Problem 44.  

Pentagon number.  Given a number, is it a pentagon ?

\begin{flalign}
P_n = \frac{n(3n-1)}{2}, \\
2*P_n = n(3n-1), \\
2*P_n = 3n^2-n, \\
0 = 3n^2-n-2*P_n, \\
\frac{1+\sqrt{1^2 - 4 * 3 * -2*P_n }}{2 * 3}, \\
\frac{1+\sqrt{1^2 + 24P_n }}{6}
\end{flalign}

Problem 50.


\newcounter{index}
\begin{flalign}
\begin{tabular}{*{20}{c|}}
 & 0 & 1 & 2 & 3 & 4 & 5 & 6 & 7 & 8 & 9 & 10 & 11 \\
\hline
Max  & 0 & 0 & 1 & 1 & 0 & 1 & 0 & 1 & 0 & 0 & 0 & 1  \\
Prev & 0 & 0 & 0 & 0 & 0 & 0 & 0 & 0 & 0 & 0 & 0 & 0  \\
\end{tabular}
\end{flalign}
Then loop through each prime starting at lowest.  2
Only update if it is a new maximum.  Keep old array
\begin{flalign}
\begin{tabular}{*{20}{c|}}
 & 0 & 1 & 2 & 3 & 4 & 5 & 6 & 7 & 8 & 9 & 10 & 11 \\
\hline
Max  & 0 & 0 & 1 & 0 & 0 & 0 & 0 & 0 & 0 & 0 & 0 & 0  \\
Prev & 0 & -1& 0& -1& -1& -1& -1& -1& -1& -1& -1& -1  \\
\end{tabular}
\end{flalign}

3
\begin{flalign}
\begin{tabular}{*{20}{c|}}
 & 0 & 1 & 2 & 3 & 4 & 5 & 6 & 7 & 8 & 9 & 10 & 11 \\
\hline
Max  & 0 & 0 & 1 & 1 & 0 & 2 & 0 & 0 & 0 & 0 & 0 & 0  \\
Prev & 0 & -1& 0& 0& -1& 2& -1& -1& -1& -1& -1& -1  \\
\end{tabular}
\end{flalign}

5
\begin{flalign}
\begin{tabular}{*{20}{c|}}
 & 0 & 1 & 2 & 3 & 4 & 5 & 6 & 7 & 8 & 9 & 10 & 11 \\
\hline
Max  & 0 & 0 & 1 & 1 & 0 & 2 & 0 & 2 & 2 & 0 & 3 & 0  \\
Prev & 0 & -1& 0& 0& -1& 2& -1& 2& 3 & -1& 5& -1  \\
\end{tabular}
\end{flalign}

\end{document}
