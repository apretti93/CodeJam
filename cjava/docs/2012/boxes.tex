\documentclass{article}
\usepackage{amsmath}
\usepackage[top=2mm, bottom=2mm, left=2mm, right=30mm]{geometry}
\begin{document}


\noindent
Modify the LCS algorithm to instead store for each row an index:value 

$(2 \rightarrow 10)$ means type 2 repeated 10 times.\vspace{5mm}

\noindent

\begin{tabular}{ l || l | c | r }
\hline
  & $(2 \rightarrow 10)$ & $(3 \rightarrow 30)$ & $(1 \rightarrow 20)$ \\
\hline
 $(1 \rightarrow 10)$ &   0 & 0 & row[40] = 1; row[49] = 10 \\
 $(2 \rightarrow 20)$ &   0 & 0 & 0 \\
 $(3 \rightarrow 25)$ &   0 & 0 & 0 \\
\end{tabular}

\vspace{5mm}Space is saved since we do not store the intermediate values.
\newline 

For the next row, we even delete the entries for row[40] and row[49] because they are redundant.
\vspace{5mm}

\begin{tabular}{ l || l | c | r }
\hline
  & $(2 \rightarrow 10)$ & $(3 \rightarrow 30)$ & $(1 \rightarrow 20)$ \\
\hline
 $(1 \rightarrow 10)$ &   0 & 0 & row[40] = 1; row[49] = 10 \\
 $(2 \rightarrow 20)$ &   row[0]=1; row[9]=10 & ... & ... \\
 $(3 \rightarrow 25)$ &   0 & 0 & 0 \\
\end{tabular}

\vspace{5mm}
Last row
\vspace{5mm}

\begin{tabular}{ l || l | c | r }
\hline
  & $(2 \rightarrow 10)$ & $(3 \rightarrow 30)$ & $(1 \rightarrow 20)$ \\
\hline
 $(1 \rightarrow 10)$ &   0 & 0 & row[40] = 1; row[49] = 10 \\
 $(2 \rightarrow 20)$ &   row[0]=1; row[9]=10 & ... & ... \\
 $(3 \rightarrow 25)$ &   row[0]=1; row[9]=10 & row[10] = 11 ; row[34] = 35 & 0 \\
\end{tabular}

\vspace{5mm}
The actual row index we don't keep track of.  
\vspace{5mm}




In the regular algorithm, if we have a match, we add 1 from top/left.

Here, we want to only use the match if it will increase the starting point for the
next matching block.

\vspace{5mm}

\begin{tabular}{ l || l | l | l | c | r }
\hline 
Column start index & 0 & 3 & 5 & 9 & 13 \\
\hline
B "string" & aaa & bb & aa & bbbb & aaa \\
\hline
Previous block row & br[0] = 0 & & br[5] = 0 \\
\hline
A element 20 x a \\
\end{tabular}


\vspace{5mm}

When matching at br[5], we add one to the pbr[4].  In the case above, that is
0.  So we can use 3 of a's length and update pr, which stores intermediary values after block rows index.

\vspace{5mm}

\begin{tabular}{ l || l | l | l | c | r }
\hline 
Column start index & 0 & 3 & 5 & 9 & 13 \\
\hline
B "string" & aaa & bb & aa & bbbb & aaa \\
\hline
Previous block row & br[0] = 0 & & br[5] = 0 \\
\hline
Previous row & pr[0] = 1 ; pr[2] = 2 & &  \\
\hline
A element 17 x a \\
\end{tabular}


\vspace{15mm}

Different example. The second block of a's, the previous block row
says we start at 5, but we could have matched 6 'a's, so we use that 
instead.

\vspace{5mm}

\begin{tabular}{ l || l | l | l | c | r }
\hline 
Column start index & 0 & 6 & 11 &  &  \\
\hline
B "string" & $6 \cdotp a$  & $5 \cdotp b$  & $8 \cdotp a$ &  &  \\
\hline
Previous block row & pbr[0] = 0 & pbr[3] = 1 ; pbr[7] = 5 &  \\
\hline
Previous row & ;  & &  \\
\hline
A element $10 \cdotp a$ \\
\end{tabular}

say the LCS matrix looked like

To build the next row, * will be $\max( lcs_{r,i+5} + 1 \| aa, lcs_{r,i+2} + 2 \| a,  lcs_{r,i+1} + 3 \| \emptyset )$


It is as if we ignored the b's entirely

Example :
5 5

A -- 3 1 4 5 3 3 5 1 2 3
B -- 2 5 3 1 4 5 4 1 4 2
\newline 



\[ \left| \begin{array}{l | l*{9}{p{16mm}} }
 & ... & c & a & a & a & b & b & a & a & a  \\
\hline 
& ... & $lcs_{0,i}$ & $lcs_{0,i+1}$ & $lcs_{0,i+2}$ & ... & ... & ... & ... & ... & ... \\
 & ... & $lcs_{1,i}$ & $lcs_{1,i+1}$ & $lcs_{1,i+2}$ & ... & ... & ... & ...  & ... & ...\\
 & ... & $lcs_{r,i}$ & $lcs_{r,i+1}$ & $lcs_{r,i+2}$ & $lcs_{r,i+3}$ & $lcs_{r,i+4}$  & $lcs_{2,i+5}$ & ... & ...  & ... \\
aaa  &   &   &   & &  & & & * &  &  \\
 \end{array} \right|\] 

\[ \begin{tabular} {l | l*{10}{p{16mm}}}
& $2 \cdotp e$ & $3 \cdotp a$ & $4 \cdotp e$ & $4 \cdotp a$ & $4 \cdotp b$  \\
\hline 
$3 \cdotp a$ & $lcs_{0,0}=1, lcs_{0,2}=3$ & $lcs_{0,i+1}$ & $lcs_{0,i+2}$ & ... & ... & ... & ... & ... & ... \\
$4 \cdotp e$ & ... & $lcs_{1,i}$ & $lcs_{1,i+1}$ & $lcs_{1,i+2}$ & ... & ... & ... & ...  & ... & ...\\
$3 \cdotp c$ & ... & $lcs_{r,i}$ & $lcs_{r,i+1}$ & $lcs_{r,i+2}$ & $lcs_{r,i+3}$ & $lcs_{r,i+4}$  & $lcs_{2,i+5}$ & ... & ...  & ... \\
$5 \cdotp a$ & ... & $lcs_{1,i}$ & $lcs_{1,i+1}$ & $lcs_{1,i+2}$ & ... & ... & ... & ...  & ... & ...\\
$2 \cdotp c$ & ... & $lcs_{1,i}$ & $lcs_{1,i+1}$ & $lcs_{1,i+2}$ & ... & ... & ... & ...  & ... & ...\\
aaa  &   &   &   & &  & & & * &  &  \\
\end{tabular} \]


\end{document}
